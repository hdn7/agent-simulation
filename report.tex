\documentclass[12pt]{article}
\usepackage{amsmath}
\usepackage{graphicx}
\usepackage{hyperref}
\usepackage[latin1]{inputenc}

\title{Simulacion de Poblados en Evoluci\'on}
\author{Hieu Do Ngoc - C411 \\ Joel David Hern\'andez Cruz - C411}

\date{22 de abril de 2019}

\begin{document}
\maketitle
\newpage

\section{Principales Ideas} 
\begin{itemize}
	\item La idea principal es programar 2 diferentes modelos de agente 
	para el robot y mediante las diferentes simulaciones a realizar, destacar
	los caracter\'isticas m\'as relevantes de cada modelo, as\'i como 
	sus diferencias entre el comportamiento de estos.
	\item Tambi\'en se extrae de las simulaciones, una comparaci\'on del 
	rendimiento de	cada robot en cuanto a sus tiempos de duraci\'on, nivel 
	de	suciedad y cantidad de veces que tuvieron ex\'ito en el trabajo o fueron
	despedidos.
	\item Para realizar las simulaciones, se generan el ambiente y los agentes
	cumpliendo con las condiciones definidas en la orientaci\'on, y se procede 
	a simular el comportamiento de los ni\~nos y el del robot utilizando uno de 
	los modelos de agente programados.	
\end{itemize}

\section{Modelos de Agentes}
\begin{itemize}
	\item \textbf{Modelo \#1}:
	El modelo de agente sigue una estrategia pro-activa con una ligera
	reactividad, 
	su comportamiento se basa en priorizar la ubicaci\'on de los ni\~nos en el 
	corral por encima de limpiar la suciedad. Cuando la suciedad se
	ha acumulado demasiado, el robot priorizar\'a entonces la 
	limpieza.\\
	En cada turno, el robot busca al ni\~no que m\'as cerca se encuentre 
	y se mueve hacia su direcci\'on para cargarlo. En el momento que carge
	a un ni\~no, busca a la casilla de corral desocupada m\'as cercana y 
	se mueve hacia este. Si en alg\'un momento, la cantidad de suciedad en el
	ambiente llega a 	niveles muy altos (cercano al 60\%), se pasa a priorizar
	la limpieza.\\
	El robot, para disminuir el nivel de suciedad simplemente encuentra la 
	casilla m\'as cercana que est\'e sucia y se mueve hacia ella para limpiarla.
	\begin{itemize} \textit{Algunos detalles sobre este modelo:}
		\item Si el ambiente llega a niveles altos de suciedad y el robot 
		est\'a cargando a un ni\~no, este lo deja en la misma casilla en que se
		encuentra para priorizar la limpieza.
		\item En el proceso de moverse hacia un ni\~no para recogerlo y guardarlo
		en el corral, si el robot se encuentra encima de una casilla sucia, este 
		no lo limpiar\'a porque prioriza por encima de todo a los ni\~nos.
		\item El robot s\'olo limpiar\'a una cierta cantidad de suciedad para que
		se deje de cumplir que el nivel de suciedad est\'a 'muy alto'. Nunca se 
		dedica limpiar toda la suciedad, excepto el caso en que todos los ni\~nos
		ya hayan sido guardados en el corral.
	\end{itemize}
	\item \textbf{Modelo \#2}:
	El segundo modelo de agente realizado sigue una estrategia similar al anterior, con solo algunas
	modificaciniones.
	Este 'toma la iniciativa' y prioriza llevar a los ni\~nos al corral y s\'olo decide limpiar cuando
	ya todos los ni\~nos hayan sido llevados al corral.\\
	Esta estrategia deber\'ia tener m\'as probabilidades de \'exito ya que se basa en la observaci\'on siguiente:
	es muy probable que los ni\~nos generan suciedades todos los turnos, mientras mas limpia sea el ambiente y/o m\'as
	'cluster' de ni\~nos hayan, mas r\'apido se llenar\'a de suciedad, y adem\'as la limpieza es bastante costosa en 
	cuanto al tiempo que demora (2 turnos por casilla).\\
	Por tanto, resulta intuitivo pensar que priorizar los generadores de la suciedad ser\'ia bastante m\'as factible que
	priorizar la limpieza de la suciedad en s\'i.\\
	La estrategia seguida por el modelo es bastante pro-activa al no reaccionar a los niveles de suciedad del ambiente 
	sin embargo, debe tener en cuenta el movimiento de los ni\~nos para encontrar y moverse hacia el ni\~no m\'as cercano,
	por lo que debe considerarse como un agente h\'ibrido. 
\end{itemize}

\section{Ideas para la implementaci\'on}
	\begin{itemize}
		\item Para programar los modelos de comportamiento del robot se crearon
		2 programas de \textit{PROLOG}, los cuales contienen hechos y reglas l\'ogicas que
		definen el comportamiento del robot para cada modelo.
		\item El una implementaci\'on en \textit{PROLOG} del m\'etodo BFS fue usado para 
		encontrar determinar caminos de menor longitud entre el robot y la suciedad,
		los ni\~nos o el corral.
		\item La representaci\'on del ambiente en cuesti\'on se hace a trav'es de
		hechos l\'ogicos en vez de lista de listas. Al generar un ambiente se
		genera din\'amicamente hechos de \textit{PROLOG}: \textit{suciedad(1,2). 
		obst\'aculo(2,4). ni\~no(5, 3). robot(0,1).}, etc.. Para lograr esto se
		utiliz\'o los meta-predicados  \textbf{dynamic}, \textbf{assert} y \textbf{retract}
		que nos ofrece \textit{PROLOG}.
	\end{itemize}
\section{Experimentaci\'on}

\end{document}

























