\documentclass[12pt]{article}
\usepackage{amsmath}
\usepackage{graphicx}
\usepackage{hyperref}
\usepackage[latin1]{inputenc}

\title{Simulacion de Poblados en Evoluci\'on}
\author{Hieu Do Ngoc - C411 \\ Joel David Hern\'andez Cruz - C411}

\date{22 de abril de 2019}

\begin{document}
\maketitle
\newpage

\section{Principales Ideas} 
\begin{itemize}
	\item La idea principal es programar 2 diferentes modelos de agente 
	para el robot y mediante las diferentes simulaciones a realizar, destacar
	los caracter\'isticas m\'as relevantes de cada modelo, as\'i como 
	sus diferencias entre el comportamiento de estos.
	\item Tambi\'en se extrae de las simulaciones, una comparaci\'on del 
	rendimiento de	cada robot en cuanto a sus tiempos de duraci\'on, nivel 
	de	suciedad y cantidad de veces que tuvieron ex\'ito en el trabajo o fueron
	despedidos.
	\item Para realizar las simulaciones, se generan los ambientes y los agentes
	de la manera especificada en la orientaci\'on, y se procede a simular el
	desarrollo del ambiente.	
\end{itemize}

\section{Modelos de Agentes}
\begin{itemize}
	\item \textbf{Modelo \#1}:
	El modelo de agente sigue una estrategia puramente reactivo, 
	su comportamiento se basa en priorizar la ubicaci\'on de los ni\~nos en el 
	corral por encima de limpiar la suciedad.\\
	En cada turno, el robot busca al ni\~no que m\'as cerca se encuentre 
	y se mueve hacia su direcci\'on para cargarlo. En el momento que carge
	a un ni\~no, busca a la casilla de corral desocupada m\'as cercana y 
	se mueve hacia este. Si en alg\'un momento, la cantidad de suciedad en el ambiente llega a 
	niveles muy altos (cercano al 60\%), se pasa a priorizar la limpieza.
	El robot, para disminuir el nivel de suciedad simplemente encuentra la 
	casilla m\'as cercana que est\'e sucia y se mueve hacia ella para limpiarla.
\end{itemize}

\section{Ideas para la implementaci\'on}
\begin{itemize}
	\item Para programar los modelos de comportamiento del robot se crearon
	2 programas de PROLOG, cada uno para un modelo, los cuales contienen 
	hechos y reglas l\'ogicas que definen el comportamiento del robot.
\end{itemize}
\section{Experimentaci\'on}

\end{document}

























